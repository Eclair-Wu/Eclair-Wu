\documentclass[11pt]{article}
\usepackage{amsmath,braket,booktabs,longtable}
\title{Eletronic Structure Theory Notes (DFT)}
\author{Yixuan Wu}
\begin{document}
\maketitle
\section{KSDFT}
\subsection{introduction}
The Kohn—Sham DFT is to replace the difficult interacting many-body system with an auxiliary system that can be solved
easily. Kohn and Sham choose a non interacting system to be the ansatz. This leads to independent-particle equations with
all the difficult many-body terms incorporated into an \textbf{exchange-correlation functional of the density}. By solving
the equations we can find the ground-state energy and density of the original interacting system with the accuracy limited
only by the approximations in the exchange-correlation functional.\\
The auxiliary hamiltonian is chosen to have kinetic operator and an effective local potential $V_{eff}^\sigma(\mathbf{r})$
acting on an electron of spin $\sigma$ at point $\mathbf{r}$. The density can be written as
\begin{equation}
    n(\mathbf{r})=\sum_{\sigma}\sum_{i=1}^{N^\sigma}{\lvert\psi_i^\sigma(\mathbf{r})\rvert}^2
\end{equation}
and the ground state energy is
\begin{equation}
    E_{KS}=T_s[n]+\int\,d\mathbf{r}V_{ext}(\mathbf{r})n(\mathbf{r})+E_{Hartree}[n]+E_{nu}+E_{xc}[n]
\end{equation}
$T_s$ is the independent-particle kinetic energy
\begin{equation}
    T_s=-\frac{1}{2}\sum_{\sigma}\sum_{i=1}^{N^\sigma}\bra{\psi_i^\sigma}\nabla^2\ket{\psi_i^\sigma}=\frac{1}{2}\sum_{\sigma}\sum_{i=1}^{N^\sigma}\int\,d\mathbf{r}{\lvert\nabla\psi_i^\sigma\rvert}^2
\end{equation}
$E_{Hartree}$ is the classical Coulomb interaction energy
\begin{equation}
    E_{Hartree}=\frac{1}{2}\int\,d\mathbf{r}d\mathbf{r^{'}}\frac{n(\mathbf{r})n(\mathbf{r^{'}})}{\lvert\mathbf{r-r^{'}}\rvert}
\end{equation}
$V_{ext}(\mathbf{r})$ is the external potential due to the nuclei and any other external fields and $E_{nu}$ is the interaction
between the nuclei.\\
$E_{xc}$ includes all many-body effects of correlation and exchange, it can be written as 
\begin{equation}
    E_{xc}=\braket{T}-T_s[n]+\braket{V_{int}}-E_{Hartree}[n]
\end{equation}
\subsection{variational equations}
The solution can be viewed as the problem of minimization with respect to either the density or the effective potential. However,
since $T_s$ depends on the orbitals and other terms depend on density, we first use the chain rule to derive the variational equation
\begin{equation}
    \frac{\delta E_{ks}}{\delta\psi_i^{\sigma*}(\mathbf{r})}=\frac{\delta T_{s}}{\delta\psi_i^{\sigma*}(\mathbf{r})}+\bigg[\frac{\delta E_{ext}}{\delta n(\mathbf{r},\sigma)}+\frac{\delta E_{Hartree}}{\delta n(\mathbf{r},\sigma)}+\frac{\delta E_{xc}}{\delta n(\mathbf{r},\sigma)}\bigg]\frac{\delta n(\mathbf{r},\sigma)}{\delta\psi_i^{\sigma*}(\mathbf{r})}=0
\end{equation}
subject to the orthonormalization constraints.\\
We can simplify the equation by
\begin{align}
    \frac{\delta T_{s}}{\delta\psi_i^{\sigma*}(\mathbf{r})}&=-\frac{1}{2}\nabla^2\psi_i^\sigma\\
    \frac{\delta n(\mathbf{r},\sigma)}{\delta\psi_i^{\sigma*}(\mathbf{r})}&=\psi_i^\sigma(\mathbf{r})
\end{align}
now the we can obatin a new equation called Kohn—Sham equation
\begin{equation}
    (H^\sigma_{KS}-\epsilon_i^\sigma)\psi_i^\sigma=0
\end{equation}
where $\epsilon_i$ is the eigenvalue and $H^\sigma_{KS}$ is the effective hamiltonian
\begin{equation}
    \begin{split}
        H^\sigma_{KS}=&-\frac{1}{2}\nabla^2+\frac{\delta E_{ext}}{\delta n(\mathbf{r},\sigma)}+\frac{\delta E_{Hartree}}{\delta n(\mathbf{r},\sigma)}+\frac{\delta E_{xc}}{\delta n(\mathbf{r},\sigma)}\\
        =&-\frac{1}{2}\nabla^2+V_{ext}(\mathbf{r})+V_{Hartree}(\mathbf{r})+V_{xc}^\sigma(\mathbf{r})\\
        =&-\frac{1}{2}\nabla^2+V_{KS}^\sigma(\mathbf{r})
    \end{split}
\end{equation}
\subsection{exchange-correlation functional}
By seperating out the kinetic energy and the long-range Hartree terms, the remaining exchange-correlation functional can be approximated as
a local or nearly local functional of density
\begin{equation}
    E_{xc}[n]=\int\,d\mathbf{r}n(\mathbf{r})\epsilon_{xc}([n],\mathbf{r})
\end{equation}
where $\epsilon_{xc}$ is an energy per electron at point $\mathbf{r}$ that depends only upon the density in some neighborhood of point $\mathbf{r}$.

\section{Basic Concepts}
\subsection{crystal}
\begin{gather}
    \text{crystal structure}=\text{Bravais lattice}+\text{basis}\\
    \text{(basis: positions and types of atom in the primitive cell)}\notag\\
    \text{space group}=\text{translation group}+\text{point group}
\end{gather}
The translation can be written as
\begin{align}
    \textbf{T(n)}=n_1\textbf{a}_1+n_2\textbf{a}_2+\cdots+n_d\textbf{a}_d\\
    \text{(d is the dimension of space)}\notag 
\end{align}
The Wigner-Seitz cell is bounded by planes that are perpendicular bisectors of the translation vectors from the central lattice point.
It's independent of the choice of the primitive translation.
It's useful because it's a unique cell defined as the set of all points closer to the central point and remain the symmtery of the Bravais
lattice.\\
The primitive vectors can be written as a saqure matrix $\mathbf{a}(a_{ij})$, where j denotes the component and i the primitive vector. And the volume
of the primitive cell is
\begin{equation}
    V=\det{\mathbf{a}}
\end{equation}
We can also defind the atom position vector $\tau_n$ where $n$ is the number of atoms in a primitive cell. We can also define it by lattice vector
\begin{equation}
    \tau_n=\sum_{i=1}^{d}\tau^l_{ni}\mathbf{a}_i
\end{equation}
\subsection{reciprocal and Brillouin zone}
Consider if we have a function defind for the crystal, it must satisfy
\begin{equation}
    f(\mathbf{r})=f(\mathbf{r+T(n)})
\end{equation}
so the function can be represented by Fourier transforms:
\begin{equation}
    \begin{split}
        f(\mathbf{q})&=\frac{1}{V_{crystal}}\int_{V_{crystal}}\,d\mathbf{r}f(\mathbf{r})e^{i\mathbf{qr}}\\
        &=\frac{1}{V_{crystal}}\sum_{n_1,n_2,\cdots}\int_{V_{cell}}\,d\mathbf{r}f(\mathbf{r})e^{i\mathbf{q(r+T(n))}}\\
        &=\frac{1}{N_{cell}}\sum_{n_1,n_2,\cdots}e^{i\mathbf{qT(n)}}\frac{1}{V_{cell}}\int_{V_{cell}}\,d\mathbf{r}f(\mathbf{r})e^{i\mathbf{qr}}
    \end{split}
\end{equation}
where ${V_{crystal}}$ is a large volume of crystal composed of $N_{cell}=N_1\times N_2\times\cdots$ cells and $\mathbf{q}$ is the wavevector.
They must satisfy the Born-Von periodic boundary conditions in each of the dimension:
\begin{gather}
    e^{i\mathbf{q}N_1\mathbf{a}_1}=e^{i\mathbf{q}N_2\mathbf{a}_2}=\cdots=1\\
    \mathbf{qa}_i=2k\pi/N_i\quad(k=integer)
\end{gather} 
and the sum over all lattice points in equation (18) vanishes for $\mathbf{q}$ except those $\mathbf{qT}=2k\pi$ for all $\mathbf{T}$, so if must
follow that $\mathbf{qa}_i=2k\pi(k=integer)$. This set of $\mathbf{q}$ is called \textbf{reciprocal lattice} and noted as $\mathbf{G}$. We define 
a new set of vectors:
\begin{equation}
    \mathbf{b}_i\mathbf{a}_1=2\pi\delta_{ij}
\end{equation}
which satisfy
\begin{equation}
    \mathbf{G(m)}=m_1\textbf{b}_1+m_2\textbf{b}_2+\cdots+m_d\textbf{b}_d
\end{equation}
and equation (18) can be written as
\begin{equation}
    f(\mathbf{q})=\frac{1}{V_{cell}}\int_{V_{cell}}\,d\mathbf{r}f(\mathbf{r})e^{i\mathbf{qr}}
\end{equation}
By the matirx notation we have
\begin{equation}
    \mathbf{b}^{T}\mathbf{a}=2\pi\mathbf{I}
\end{equation}
and the volume of the reciprocal lattice, also called the first Brillouin zone is
\begin{equation}
    V_{BZ}=\det{\mathbf{b}}=\frac{{(2\pi)}^d}{V_{cell}}
\end{equation}
Inmitating the defination of $\tau$, we can define the vector in reciprocal space
\begin{equation}
    k_n=\sum_{i=1}^{d}k_{ni}^l\mathbf{b}_i
\end{equation}
in matrix form it becomes
\begin{equation}
    \mathbf{k}=\mathbf{k}^l\mathbf{b}
\end{equation}
To find the nearest neighbors in real space or the lowest components in the reciprocal space within a cutoff radius, $\mathbf{T}(n_1,n_2,n_3)$ is
stricted by $N_1,N_2,N_3$:
\begin{gather}
    -N_i \leq  n_i\leq N_i \quad(i=1,2,3)\\
    N_i=\frac{\lvert\mathbf{b}_i\rvert}{2\pi}R_{max}\quad(i=1,2,3)\\
    M_i=\frac{\lvert\mathbf{a}_i\rvert}{2\pi}G_{max}\quad(i=1,2,3)
\end{gather}
\subsection{Excitations and the Bloch theorem}
The above discussion is all about the periodic function in crystal, but excitations of the crystal don't have periodicity.
Consier an operator defined for the periodic system, it must be invariant to any $\mathbf{T(n)}$. The signle-particle hamiltonian
$\hat{H}$ is an example. We define a translation operator $\hat{\mathbf{T}}_n$
\begin{equation}
    \hat{\mathbf{T}}_n\psi(\mathbf{r})=\psi(\mathbf{r+T(n)})=t_n\psi(\mathbf{r})
\end{equation}
and it commutes with the hamiltonian
\begin{equation}
    \hat{\mathbf{T}}_n\hat{\mathbf{H}}=\hat{\mathbf{H}}\hat{\mathbf{T}}_n
\end{equation}
Now we talk about the eigenvalue of the operators
\begin{gather}
    \hat{\mathbf{T}}_{\mathbf{n}_1}\hat{\mathbf{T}}_{\mathbf{n}_2}=\hat{\mathbf{T}}_{\mathbf{n}_1+\mathbf{n}_2}\\
    \hat{\mathbf{T}}_{\mathbf{n}_1+\mathbf{n}_2}\psi(\mathbf{r})=t_{\mathbf{n}_1}t_{\mathbf{n}_2}\psi(\mathbf{r})
\end{gather} 
we break the translation into the product of the primitive translations
\begin{gather}
    t_{\mathbf{n}}={t(\mathbf{a}_1)}^{n_1}{t(\mathbf{a}_2)}^{n_2}\cdots
\end{gather} 
since the $t(\mathbf{a})$ must be unity and satisfy the boundary condition
\begin{gather}
    t(\mathbf{a})=e^{i2\pi y_i} \qquad {t(\mathbf{a})}^{N_i}=1
\end{gather}
we can derive
\begin{gather}
    t_{\mathbf{n}}=e^{i\mathbf{kT_n}}\\
    \mathbf{k}=\sum_{i=1}^{d}\frac{k_i}{N_i}\mathbf{b}_i
\end{gather}
Now we have the Bloch theorem
\begin{equation}
    {\mathbf{T}}_n\psi(\mathbf{r})=e^{i\mathbf{kT_n}}\psi(\mathbf{r})
\end{equation}
the eigenstates of any periodic hamiltonian can be chosen with definite values of $\mathbf{k}$, and the eigenfunction is
\begin{equation}
    \psi_{\mathbf{k}}(\mathbf{r})=e^{i\mathbf{kr}}u_{\mathbf{k}}(\mathbf{r})
\end{equation}
so the hamiltonian equation can be written as
\begin{equation}
    e^{-i\mathbf{kr}}\hat{H}e^{i\mathbf{kr}}u_{i,{\mathbf{k}}}(\mathbf{r})=\epsilon_{i,k}u_{i,{\mathbf{k}}}(\mathbf{r})
\end{equation}
In the limit of a large crystal, the spacing of the $\mathbf{k}$ points goes to zero and $\mathbf{k}$ can be considered a continuous variable. For 
each $\mathbf{k}$ there is a discrete set of eigenstates $\{\epsilon_{i,k}\}$ which leads to band of eigenvalues.\\
For many properties, it's essential to sum over the states labeled by $\mathbf{k}$. If we choose the eigenfunctions in a large crystal, there is exactly
one value $\mathbf{k}$ for each cell, and the average value per cell can becomes
\begin{equation}
    \bar{f_i}=\frac{1}{N_k}\sum_{\mathbf{k}}f_i(\mathbf{k})
\end{equation}
where i denotes any of the discrete set of states at each $\mathbf{k}$. By taking the limit of a continuous variable
in Fourier space with a volume per $\mathbf{k}$ point of $\frac{V_{BZ}}{N_k}$, the sum is converted to an integral
\begin{equation}
    \bar{f_i}=\frac{1}{V_{BZ}}\int_{BZ}\,d\mathbf{k}f_i(\mathbf{k})=\frac{V_{cell}}{{2\pi}^d}\int_{BZ}\,d\mathbf{k}f_i(\mathbf{k})
\end{equation}
There is an additional symmtery in the system with no magnetic field, the hamiltonian is invariant to time reversal.
So we need to calculate states at both $\mathbf{k}$ and $-\mathbf{k}$ in crystal, $\psi_{i,-\mathbf{k}}$ can always be
chosen  $\psi^*_{i,-\mathbf{k}}$.
\subsection{point symmtery}
Now we can consider the point symmteries which include rotations, inversions reflections and some non-symmorphic operations.
Such operations form a group$\{R_n,n=1,\ldots,N_{group}\}$, their operation on any function of the system such as electron density is
\begin{equation}
    R_n g(\mathbf{r})=g(R_n\mathbf{r}+\mathbf{t}_n)
\end{equation}
where $R_n$ is the symmorphic operation and $\mathbf{t}_n$ is the non-symmorphic operation. By applying them to the SE it leads to a
new equation $\mathbf{r}\rightarrow R_n\mathbf{r}+\mathbf{t}_n$ and $\mathbf{k}\rightarrow R_n\mathbf{k}$ because $\mathbf{t}_n$ has no
effect on reciprocal space.
\begin{equation}
    \psi^{R_i\mathbf{k}}(R_i\mathbf{r}+\mathbf{t}_i)=\psi^{\mathbf{k}}(\mathbf{r})
\end{equation}
They have the same eigenvalue $\epsilon_{i}^{\mathbf{k}}$.\\
At high symmtery k points $(\mathbf{k}\equiv R_n\mathbf{k})$, the eigenvectors can be classified according to the group representations. And
we can define the irreducible Brillouin zone which is the smallest fraction of the BZ that is sufficient to determine all the information on
the excitations of the crystal. All other k points outside the IBZ are related by the symmtery operations.



































































































\end{document}